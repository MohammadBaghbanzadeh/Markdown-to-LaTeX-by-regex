\documentclass{article}
\usepackage{graphicx} % Required for inserting images

\title{Wikipedia to LaTeX using regex}
\author{Mohammad Baghbanzadeh}
\date{April 2023}

\begin{document}

\maketitle


==Origin==
{{quote | text={{lang|de|...es ist sehr wahrscheinlich, dass alle Wurzeln reell sind. Hiervon wäre allerdings ein strenger Beweis zu wünschen; ich habe indess die Aufsuchung desselben nach einigen flüchtigen vergeblichen Versuchen vorläufig bei Seite gelassen, da er für den nächsten Zweck meiner Untersuchung entbehrlich schien.}}<br /><br />...it is very probable that all roots are real. Of course one would wish for a rigorous proof here; I have for the time being, after some fleeting vain attempts, provisionally put aside the search for this, as it appears dispensable for the immediate objective of my investigation.|source=Riemann's statement of the Riemann hypothesis, from {{harv|Riemann|1859}}. (He was discussing a version of the zeta function, modified so that its roots (zeros) are real rather than on the critical line.)}}

Riemann's original motivation for studying the zeta function and its zeros was their occurrence in his [[Explicit formulae (L-function)|explicit formula]] for the [[prime-counting function|number of primes]] {{pi}}(	extit{z}) less than or equal to a given number 	extit{z}, which he published in his 1859 paper "[[On the Number of Primes Less Than a Given Magnitude]]". His formula was given in terms of the related function

:$\Pi(x) = \pi(x) + \tfrac{1}{2} \pi(x^{\frac{1}{2}}) +\tfrac{1}{3} \pi(x^{\frac{1}{3}}) + \tfrac{1}{4}\pi(x^{\frac{1}{4}}) + \tfrac{1}{5} \pi(x^{\frac{1}{5}}) +\tfrac{1}{6}\pi(x^{\frac{1}{6}}) +\cdots $

which counts the primes and prime powers up to 	extit{z}, counting a prime power 	extit{z}$	extit{z}$ as {{frac|1|	extit{z}}}. The number of primes can be recovered from this function by using the [[Möbius inversion formula]],

:$\begin{align}
\pi(x) &= \sum_{n=1}^{\infty}\frac{\mu(n)}{n}\Pi(x^{\frac{1}{n}}) \\
       &= \Pi(x) -\frac{1}{2}\Pi(x^{\frac{1}{2}}) - \frac{1}{3}\Pi(x^{\frac{1}{3}}) - \frac{1}{5}\Pi(x^{\frac{1}{5}}) + \frac{1}{6} \Pi(x^{\frac{1}{6}}) -\cdots,
\end{align}$

where 	extit{z} is the [[Möbius function]]. Riemann's formula is then

:$\Pi_0(x) = \operatorname{li}(x) - \sum_\rho \operatorname{li}(x^\rho) -\log 2 + \int_x^\infty\frac{dt}{t(t^2-1)\log t}$

where the sum is over the nontrivial zeros of the zeta function and where Π$0$ is a slightly modified version of Π that replaces its value at its points of [[Discontinuity (mathematics)|discontinuity]] by the average of its upper and lower limits:

:$\Pi_0(x) = \lim_{\varepsilon \to 0}\frac{\Pi(x-\varepsilon) + \Pi(x+\varepsilon)}2. $

The summation in Riemann's formula is not absolutely convergent, but may be evaluated by taking the zeros ρ in order of the absolute value of their imaginary part. The function li occurring in the first term is the (unoffset) [[logarithmic integral function]] given by the [[Cauchy principal value]] of the divergent integral

:$\operatorname{li}(x) = \int_0^x\frac{dt}{\log t}.$

The terms li(	extit{z}$	extit{z}$) involving the zeros of the zeta function need some care in their definition as li has branch points at 0 and 1, and are defined (for 	extit{z}&nbsp;>&nbsp;1) by analytic continuation in the complex variable 	extit{z} in the region Re(	extit{z})&nbsp;>&nbsp;0, i.e. they should be considered as {{nowrap|[[Exponential integral|Ei]](	extit{z} log 	extit{z})}}. The other terms also correspond to zeros: the dominant term li(	extit{z}) comes from the pole at 	extit{z}&nbsp;=&nbsp;1, considered as a zero of multiplicity −1, and the remaining small terms come from the trivial zeros. For some graphs of the sums of the first few terms of this series see {{harvtxt|Riesel|Göhl|1970}} or {{harvtxt|Zagier|1977}}.

This formula says that the zeros of the Riemann zeta function control the [[oscillation]]s of primes around their "expected" positions. Riemann knew that the non-trivial zeros of the zeta function were symmetrically distributed about the line {{nowrap|	extit{z} {{=}} 1/2 + 	extit{z},}} and he knew that all of its non-trivial zeros must lie in the range {{nowrap|0 ≤ Re(	extit{z}) ≤ 1.}} He checked that a few of the zeros lay on the critical line with real part 1/2 and suggested that they all do; this is the Riemann hypothesis.

{{quote box|quote=The result has caught the imagination of most mathematicians because it is so unexpected, connecting two seemingly unrelated areas in mathematics; namely, [[number theory]], which is the study of the discrete, and [[complex analysis]], which deals with continuous processes. {{harv|Burton|2006|p=376}}}}
{{clear}}

==Consequences==
The practical uses of the Riemann hypothesis include many propositions known to be true under the Riemann hypothesis, and some that can be shown to be equivalent to the Riemann hypothesis.

===Distribution of prime numbers===
[[Riemann's explicit formula]] for [[prime-counting function|the number of primes less than a given number]] in terms of a sum over the zeros of the Riemann zeta function says that the magnitude of the oscillations of primes around their expected position is controlled by the real parts of the zeros of the zeta function. In particular the error term in the [[prime number theorem]] is closely related to the position of the zeros. For example, if β is the [[upper bound]] of the real parts of the zeros, then <ref>{{harvtxt|Ingham|1932}}, Theorem 30, p. 83; {{harvtxt|Montgomery|Vaughan|2007}}, p. 430</ref>
$\pi(x) - \operatorname{li}(x) = O \left( x^{\beta} \log x \right).$
It is already known that 1/2 ≤ β ≤ 1.{{sfnp|Ingham|1932|p=82}}

[[#CITEREFvon Koch1901|Von Koch (1901)]] proved that the Riemann hypothesis implies the "best possible" bound for the error of the prime number theorem. A precise version of Koch's result, due to {{harvtxt|Schoenfeld|1976}}, says that the Riemann hypothesis implies

:$|\pi(x) - \operatorname{li}(x)| < \frac{1}{8\pi} \sqrt{x} \log(x), \qquad \text{for all } x \ge 2657,$
where $\pi(x)$ is the [[prime-counting function]], $\operatorname{li}(x)$ is the [[logarithmic integral function]], and $\log(x)$ is the [[natural logarithm]] of 	extit{z}.

{{harvtxt|Schoenfeld|1976}} also showed that the Riemann hypothesis implies

:$|\psi(x) - x| < \frac{1}{8\pi} \sqrt{x} \log^2 x, \qquad \text{for all } x \ge 73.2, $

where $\psi(x)$ is [[Chebyshev function|Chebyshev's second function]].

{{harvtxt|Dudek|2014}} proved that the Riemann hypothesis implies that for all $x \geq 2$ there is a prime $p$ satisfying
:$x - \frac{4}{\pi} \sqrt{x} \log x < p \leq x$.
This is an explicit version of a theorem of [[Cramér]].

===Growth of arithmetic functions===
The Riemann hypothesis implies strong bounds on the growth of many other [[arithmetic function]]s, in addition to the primes counting function above.

One example involves the [[Möbius function]] μ. The statement that the equation

:$\frac{1}{\zeta(s)} = \sum_{n=1}^\infty \frac{\mu(n)}{n^s}$

is valid for every 	extit{z} with real part greater than 1/2, with the sum on the right hand side converging, is equivalent to the Riemann hypothesis. From this we can also conclude that if the [[Mertens function]] is defined by

:$M(x) = \sum_{n \le x} \mu(n)$

then the claim that

:$M(x) = O\left(x^{\frac{1}{2}+\varepsilon}\right)$

for every positive ε is equivalent to the Riemann hypothesis ([[John Edensor Littlewood|J.E. Littlewood]], 1912; see for instance: paragraph 14.25 in {{harvtxt|Titchmarsh|1986}}). (For the meaning of these symbols, see [[Big O notation]].) The [[determinant]] of the order 	extit{z} [[Redheffer matrix]] is equal to 	extit{z}(	extit{z}), so the Riemann hypothesis can also be stated as a condition on the growth of these determinants. Littlewood's result has been improved several times since then, by [[Edmund Landau]],<ref>{{Citation | last1=Landau | first1=Edmund | author1-link=Edmund Landau | title=Über die Möbiussche Funktion | year=1924 | journal=Rend. Circ. Mat. Palermo | volume=48 | pages=277–280}}</ref> [[Edward Charles Titchmarsh]],<ref>{{Citation | last1=Titchmarsh | first1=Edward Charles | title=A consequence of the Riemann hypothesis | year=1927 | journal=J. London Math. Soc. | volume=2 | pages=247–254}}</ref> Helmut Maier and [[Hugh Lowell Montgomery|Hugh Montgomery]],<ref>{{Citation | last1=Maier | first1=Helmut | last2=Montgomery | first2=Hugh | title=The sum of the Möbius function | year=2009 | journal=Bull. London Math. Soc. | volume=41 | pages=213–226 | doi=10.1112/blms/bdn119}}</ref> and [[Kannan Soundararajan]].<ref>{{Citation | last1=Soundararajan | first1=Kannan | author1-link=Kannan Soundararajan | title=Partial sums of the Möbius function | journal=J. Reine Angew. Math. | volume=631 | year=2009 | pages=141–152 | doi=10.1515/CRELLE.2009.044}}</ref>  Soundararajan's result is that
:$M(x) = O(x^{1/2}\exp\big((\log x)^{1/2}(\log \log x)^{14}\big).$
The Riemann hypothesis puts a rather tight bound on the growth of 	extit{z}, since {{harvtxt|Odlyzko|te Riele|1985}} disproved the slightly stronger [[Mertens conjecture]]

:$|M(x)| \le \sqrt x.$

Another closely related result is due to {{harvtxt|Björner|2011}}, that the Riemann hypothesis is equivalent to the statement that the [[Euler characteristic]] of the [[simplicial complex]] determined by the lattice of integers under divisibility is $o(n^{1/2+\epsilon})$ for all $\epsilon>0$ (see [[incidence algebra]]).

The Riemann hypothesis is equivalent to many other conjectures about the rate of growth of other arithmetic functions aside from μ(	extit{z}). A typical example is [[Robin's theorem]],{{sfnp|Robin|1984}} which states that if σ(	extit{z}) is the [[divisor function|sigma function]], given by

:$\sigma(n) = \sum_{d\mid n} d$

then

:$\sigma(n) < e^\gamma n \log \log n$

for all 	extit{z} > 5040 if and only if the Riemann hypothesis is true, where γ is the [[Euler–Mascheroni constant]].

A related bound was given by [[Jeffrey Lagarias]] in 2002, who proved that the Riemann hypothesis is equivalent to the statement that:
:$ \sigma(n) < H_n + \log(H_n)e^{H_n}$
for every [[natural number]] 	extit{z} &gt; 1, where $H_n$ is the 	extit{z}th [[harmonic number]].<ref>{{Citation | last1=Lagarias | first1=Jeffrey C. | author1-link=Jeffrey C. Lagarias | title=An elementary problem equivalent to the Riemann hypothesis | doi=10.2307/2695443 | jstor=2695443 | mr=1908008 | year=2002 | journal=[[American Mathematical Monthly|The American Mathematical Monthly]] | issn=0002-9890 | volume=109 | issue=6 | pages=534–543| arxiv=math/0008177 | s2cid=15884740 }}</ref>

The Riemann hypothesis is also true if and only if the inequality
:$\frac{n}{\varphi (n)}<e^\gamma \log\log n+\frac{e^\gamma (4+\gamma-\log 4\pi)}{\sqrt{\log n}}$
is true for all 	extit{z} ≥ 	extit{z}$120569$# where φ(	extit{z}) is [[Euler's totient function]] and 	extit{z}$120569$# is the [[Primorial|product of the first]] 120569 primes.{{sfnp|Broughan|2017|loc=Corollary 5.35}}

Another example was found by [[Jérôme Franel]], and extended by [[Edmund Landau|Landau]] (see {{harvtxt|Franel|Landau|1924}}). The Riemann hypothesis is equivalent to several statements showing that the terms of the [[Farey sequence]] are fairly regular. One such equivalence is as follows: if 	extit{z}$	extit{z}$ is the Farey sequence of order 	extit{z}, beginning with 1/	extit{z} and up to 1/1, then the claim that for all ε > 0

:$\sum_{i=1}^m|F_n(i) - \tfrac{i}{m}| = O\left(n^{\frac{1}{2}+\epsilon}\right)$

is equivalent to the Riemann hypothesis. Here

:$m = \sum_{i=1}^n\phi(i)$

is the number of terms in the Farey sequence of order 	extit{z}.

For an example from [[group theory]], if 	extit{z}(	extit{z}) is [[Landau's function]] given by the maximal order of elements of the [[symmetric group]] 	extit{z}$	extit{z}$ of degree 	extit{z}, then {{harvtxt|Massias|Nicolas|Robin|1988}} showed that the Riemann hypothesis is equivalent to the bound

:$\log g(n) < \sqrt{\operatorname{Li}^{-1}(n)}$


\end{document}

